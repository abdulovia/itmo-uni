\documentclass[14pt]{extreport}
\usepackage{gost}

\begin{document}

\tableofcontents

\chapter{Часть I Объятия}

\section{Глава 1 Как относится Кастро к России}
В глазах Америки приемлемость режима в латиноамериканской стране определяется степенью его антикоммунизма.

В то время как американские журналисты, официальные лица и студенты не подвергали сомнению заверения Кастро о независимости от международного коммунизма, Рауль Кастро упорно добивался революционных изменений в отношениях между Москвой и Гаваной. Президиум ЦК под руководством Хрущева по просьбе Рауля поручил Международному отделу ЦК КПСС, занимавшемуся отношениями с компартиями других стран, Министерству обороны и КГБ организовать отправку двух выпускников советских военных академий на Кубу.

Первый серьезный контакт между кубинскими повстанцами и Кремлем был косвенным. В декабре 1958 года в посольстве Чехословакии в Мексике обсуждали вопрос "поставок стрелкового вооружения, пушек и обмундирования для повстанческих батальонов Фиделя Кастро". Кремль, уверенный в том, что обязан при любых обстоятельствах поддерживать революционные силы в мире, 27 декабря 1958 года одобрил "намерение чешских друзей помочь освободительному движению на Кубе".

\section{Глава 2 Наш человек в Гаване}

В конце своего пребывания в Калифорнии Хрущев в менее эмоциональной форме с горечью поведал дипломату США Лоджу о том, какие мысли возникли у него в связи с посещением США. "Советский Союз никогда не отрицал, что в США высокий жизненный уровень и наиболее эффективные методы производства. Именно поэтому СССР выбрал Вашу страну как достойного оппонента и конкурента". 

В конце сентября 1959 года Хрущев, сильно озабоченный экономическим отставанием Советского Союза, возвратился в Москву. Президиум ЦК в это время решал сложную проблему: следует ли социалистическому лагерю бросить вызов американскому колоссу в Карибском бассейне. Кубинцы сделали попытку закупить оружие у Польши, и Варшава запросила руководящих указаний у Москвы. 

Многие в Кремле посчитали необходимым пересмотреть возможность отношений с Фиделем Кастро. Если попытаться определить момент, когда США и СССР начали скатываться к первому прямому военному столкновению, то это был день в конце сентября 1959 года. 

Президиум ЦК под председательством Хрущева проголосовал за "одобрение решения Польши снабдить Кубу некоторыми видами стрелкового оружия, изготовленного на польских заводах по советским лицензиям". Одобрив продажу оружия Кубе, Хрущев взял на себя риск советского вмешательства в Латинской Америке.

"Товарищ Хрущев был рад, что ему удалось то, чего не смог сделать Сталин - проникнуть в Латинскую Америку. Во-первых, проникновение в латиноамериканский регион не было целью нашей политики, а во-вторых, это означало, что наша страна должна была взять на себя обязательства осуществлять военные поставки за океан на расстояние 15 000 км". Но до этого признания ошибки сверхдержавы оказались на пороге войны.

% Прибытие Алексеева
Советское руководство было слабо осведомлено о Кубе Кастро. Даже когда в январе 1959 года Ф.Кастро триумфально вступил в Гавану, Кремль располагал только информацией, представленной социалистической партией. В феврале 1959 года советская разведка задействовала опытного сотрудника, который уже проявил большое искусство, войдя в доверие политической элиты Аргентины и Бразилии.

Александр Алексеев – советский дипломат, Чрезвычайный и полномочный посол, сотрудник службы внешней разведки, полковник – был отозван из Аргентины. Именно КГБ направил его в Гавану со специальной миссией: установить контакт с высшими руководителями Кубы. 

Когда Алексеев прибыл, Фидель и особенно Рауль Кастро сразу увидели, что это не просто журналист, а представитель определенного ведомства. Они установили с ним доверительные отношения. Когда им что-либо было нужно, то они чаще обращались прямо к Алексееву, чем к послу. Алексеев сейчас же связывался с Центром и сообщал нам о нуждах Кубы.

Встречи с Алексеевым, рост авторитета Кастро, слабость оппонентов в стране и в США осенью 1959 года меняли отношения левого крыла кубинской революции к СССР. После возвращения из-за границы Че Гевара быстро занял прежнее положение основного посредника между Кастро и социалистической партией.

Фидель поставил своего брата Рауля во главе всех силовых структур Кубы.

%Микоян и Кастро

Для Кастро приезд Микояна на Кубу символизировал движение кубинской революции по социалистическому пути. Революционные настроения Кастро и его сторонников вскоре получили широкую огласку и стали основой сближения с Советским Союзом.

Микоян был первым представителем советского руководства, которому поступали подробные отчеты о ситуации на Кубе. В ноябре 1959 года Микоян встретился с Алексеевым на советской выставке в Мехико. В течение первого месяца пребывания на Кубе Алексеев не направлял никаких сообщений в Москву, так как КГБ, считая, что он может передавать важную информацию через резидентуру в Мехико, не обеспечил его надежными каналами связи. Алексеев приурочил первый приезд в Мехико к советской выставке, чтобы передать члену Президиума ЦК новые данные о ситуации на революционной Кубе.

Микоян был доволен сообщением Алексеева. Офицер КГБ привык к более серьезным делам, чем представление информации политическим руководителям.

Микоян внес предложения Алексеева предложения в свой доклад Москве; они легли в основу отношений с революционной Кубой. Во-первых, на последующие пять лет Кремль должен подписать с Кубой договор о ежегодной поставке в СССР 500 000-600 000 тонн сахара. Он рекомендовал бартерное соглашение.

Биографы Кастро спорят по поводу того, понимал ли он неизбежность сближения с Советским Союзом. Это утверждение основано на тезисе напряженных отношений между НСП и Фиделем Кастро.

\section{Глава 3 Ля Кубр}

Кастро не мог забыть инцидент "Ля Кубр" в 1960. Он заявил, что абсолютно уверен, взрыв французского парохода произведен американцами. Несмотря на отсутствие юридических доказательств, Фидель был убежден, что США выкручивают руки своим союзникам, стремясь не допустить поставок оружия на Кубу, и саботаж на судне - это предупреждение.

Взрыв в гаванском порту послужил поводом для усиления антиамериканской риторики. Кастро, однако, считал, что еще не пришло время для официального объявления о восстановлении дипломатических отношений с СССР. 

Заявление Хрущева о том, что СССР готов применить ядерное оружие для защиты Кубы от США означала прорыв в советско-кубинских отношениях. Символом укрепления связей явился визит Рауля Кастро в Москву и его встреча с Хрущевым. Советский Союз считал, что сделал все возможное для поддержки Кубы.

И Кастро, и Хрущев затеяли игру, и, похоже, она стоила свеч. Куба выбрала социалистический путь развития. 31 июля Алексеев телеграфировал: "Фидель Кастро выражает глубокую признательность советскому правительству и лично Н.С.Хрущеву за выполнение всех просьб по поставке оружия".

\section{Глава 4 Куба -да, янки - нет!}

2 сентября харизматический Кастро выступил на миллионном митинге перед кубинским народом.

Фидель грозил США кулаком. "Мы национализировали многие североамериканские компании, но некоторые держим в резерве. Если они продолжат экономическую агрессию против нас, мы национализируем все их предприятия".

Революционный долг, заявил Кастро, бороться за провозглашение права как в своей стране, так и за рубежом. "Мы подтверждаем веру в то, что Латинская Америка освободится от ига империализма янки, несмотря на решения карманных министров иностранных дел" По словам Кастро, Куба слышит голоса угнетенных латиноамериканцев "Мы здесь! - воскликнул Кастро. - Куба вас никогда не оставит!

Уверенное в поддержке СССР, кубинское руководство начало публиковать свои планы проведения социалистической революции на Кубе. В начале октября кубинцы собирались завершить национализацию ключевых отраслей экономики.

Кеннеди сделал кубинскую проблему главным пунктом своей избирательной кампании.

В речи 9 ноября Фидель Кастро говорил, что ядерный зонтик над Кубой мог бы стать сдерживающим фактором в кризисных ситуациях. Кастро понимал, что без помощи Москвы Куба не сможет защитить себя от вероятного американского вторжения.

В речи 6 января 1961 года Хрущев провозгласил амбициозный курс советской внешней политики, заявив, что приветствует Фиделя Кастро как законного члена социалистического блока. С января 1961 года Хрущев связывал свое лидерство в коммунистическом мире и престиж Советского Союза с успехами Кубы и Кастро.

\chapter{Часть II Столкновение}

\section{Глава 1 Залив Кочинос}

В 1961 Джон Фицджеральд Кеннеди становится самым молодым в истории США президентом.

Операция в бухте Кочинос — военная операция, организованная в апреле 1961 года при участии правительства США с целью свержения правительства Фиделя Кастро на Кубе. Операция была прервана огневой мощью на крошечном береговом плацдарме.

Залив Кочинос привел к наихудшему для Кеннеди Результату: к неуязвимой Кубе в Карибском бассейне, не желающей вмешательства извне. Он получил коммунистическое государство в восьми минутах лета от США.

Вопрос, который после личной неудачи Кеннеди возникал у многих: смогут ли США примириться с существованием советского плацдарма на своем заднем дворе, достигнув приемлемой договоренности с СССР. Не только судьба шести миллионов кубинцев, но и характер соперничества сверхдержав зависели от ответа на этот вопрос.


\section{Глава 2 Урок президенту}

На Венском саммите – встрече Хрущева и Кеннеди 4 июня 1961, Кеннеди признал, что нападение на Кубу было ошибкой, но неодобрительно отозвался о новой доктрине Хрущева поддержки "священных войн". Символом провала американской политики стал крах операции в Заливе Кочинос. Хрущев предостерег Кеннеди, что политика США в отношении Кубы "чревата серьезной опасностью", и выразил пожелание, чтобы конфликт между США и Кубой был решен путем "мирного соглашения".

По вопросу запрещения ядерных испытаний серьезных дискуссий между Хрущевым и Кеннеди тоже не получилось. Кеннеди понял, что Хрущев не намерен вести никаких серьезных обсуждений, пока не добьется своих целей по берлинскому вопросу.

Венский саммит сильно разочаровал Кеннеди. Он считал, что сделал все возможное для улучшения климата во взаимоотношениях двух сверхдержав.


\section{Глава 3 Кондор и Мангуста}

Кубинский проект (также известный как операция «Мангуст») был разработкой Центрального разведывательного управления США в первые годы пребывания в должности президента США Джона Ф. Кеннеди. На 30 ноября 1961 агрессивные секретные операции против коммунистического правительства Фиделя Кастро на Кубе были санкционированы президентом Кеннеди. Операцию возглавил генерал ВВС Эдвард Лансдейл после неудачного вторжения в заливе Свиней.

% В слайде
Операция «Мангуст» была секретной программой пропаганды, психологической войны и диверсии против Кубы для отстранения коммунистов от власти. Программа была разработана в администрации Кеннеди; Американские политики хотели видеть «новое правительство, с которым Соединённые Штаты смогут жить в мире».


\section{Глава 4 Трудности в Тропиках}

В начале 1962 года Фидель Кастро столкнулся с трудностями, отнюдь не связанными с операцией "Мангуста". В январе и в феврале на Кубе не хватало продовольствия, чтобы удовлетворить основные потребности людей. Сельскохозяйственное производство было в упадке, а на складах по всей стране почти не оставалось запасов.

Экономические трудности заставили Кастро предпринять решительные шаги. Прежде всего он реорганизовал Институт аграрной реформы, который должен был служить витриной новой Кубы.	

\section{Глава 5}

Май 1962 года отличался от октября 1960 года или даже февраля 1962 года, когда Хрущев, возможно впервые, говорил, что Джон Кеннеди готов вторгнуться на Кубу, Две главные проблемы для Советского Союза - советско-американские отношения и будущее сотрудничества с Фиделем Кастро - столкнулись - столкнулись в этом месяце, мае 1962 года. Решение Кеннеди возобновить ядерные испытания в апреле 1962 года не было неожиданным после решения Москвы в одностороннем порядке нарушить мораторий в августе 1961 года. Но для Хрущева, который получал тревожные сигналы из Пентагона, это действие Кеннеди было знаком вновь обретенных американцами агрессивных намерений. Отсутствие прогресса на переговорах по Берлину, усиление американской активности в Юго-Восточной Азии и на фоне этого изменения в стратегическом плане - все это приобретало угрожающие размеры вызова, который Хрущев не мог игнорировать.

% В слайде
С момента приезда в Гавану Алексеева осенью 1959 года Кремль вел на острове двойственную политику. Укрепляя отношения с Фиделем Кастро, Москва в то же время стремилась продвигать влиятельных коммунистов в его окружении.

Эти причины поздней весной 1962 года побудили Хрущева решиться на смелый шаг, чтобы напомнить Вашингтону о советской мощи и обеспечить Кремлю то уважение американцев, которого он заслуживал. Одновременно Хрущев хотел продемонстрировать Кастро лично и решительно, что Советский Союз защитит революцию По мнению Хрущева, - а он принял это решение практически единолично, - советская ядерная база на Кубе была единственным способом дать ответ на все трудные проблемы одновременно.

\section{Глава 6}

Для Хрущева это был напряженный период ожидания. Согласится ли президент Соединенных Штатов терпеть ракеты? В конце концов, именно на это пришлось пойти Хрущеву в связи с американскими ракетами в Турции. Настало время, как часто говорил Хрущев, чтобы США почувствовали на себе психологическую цену взаимного сдерживания. Будучи реалистом, Хрущев в то же время понимал, что ответ на вопрос: "Что будет делать Кеннеди?" - осложнялся особенностями американской политической системы. Кеннеди возможно бы и понял необходимость примириться с советскими ракетами на Кубе, но он почти неизбежно столкнулся бы с решительным противодействием "реакционных сил" в Пентагоне и со стороны республиканской партии. 


\section{Глава 7 Теперь мы можем поддать вам}

До 7 сентября 1962 года в Кремле планировали развернуть на Кубе лишь один вид тактических ядерных ракет - береговые батареи крылатых ракет ФКР. В принципе тактические ракеты разработаны для применения в ходе войны, а не для отражения агрессии. Первоначально Хрущев хотел защитить Кубу, превратив остров в стратегический форпост. Но сейчас казалось, что, возможно, Советскому Союзу придется оборонять Кубу. 
Если бы Кеннеди принял решение напасть на Кубу, а Хрущев дал бы зеленый свет своим командирам на применение тактических ядерных снарядов, то Советский Союз впервые применил бы ядерное оружие в начале новой войны.

Этим решением Хрущев подтвердил, что контролирует стратегию в Карибском бассейне, и одновременно подчеркнул ядерные обязательства по отношению к Фиделю Кастро. 

Когда закончился сентябрь 1962 года, Кеннеди и Хрущев оказались гораздо ближе к военным действиям, чем им этого хотелось бы. С момента вступления в должность президента Кеннеди полагал, что ему удастся устранить Кастро, не посылая на Кубу американский десант. Он и его брат поощряли ЦРУ в том, чтобы использовать любые средства, возможно включая и убийство, дабы устранить от власти братьев Кастро и Че Гевару. Этот подход не оправдал себя. Теперь Кеннеди рассматривал военную операцию, о которой мечтали его подчиненные после поражения в Заливе Кочинос в 1961 году.

\section{Глава 8}

Президент США Кеннеди хотел направить ВВС на Кубу без предупреждения, но опасался, что "шок союзников по НАТО будет роковым". Но даже при этом он мог бы пойти на риск, если бы был уверен в полном уничтожении ракет. Но разведка США продолжала обнаруживать пусковые комплексы, и, несмотря на регулярные полеты разведывательных самолетов США над Кубой, Кеннеди не был уверен, что обнаружены все ракеты. Теперь настало время для решений Хрущева. Если советский лидер не боится угрозы войны, Кеннеди может первым перейти грань. Тогда, сказал Кеннеди себе и Исполкому, "дело можно завершить только вторжением"


\section{Глава 9 Ракетный кризис}

Для ответа на блокаду США и письмо Кеннеди, призывающее к "благоразумию", собрался Президиум. В отличие от Кеннеди, который 23 октября создал специальную группу, так называемый Исполком, для поиска выхода из кризиса, Хрущев не сделал этого. В КГБ на Лубянке работала антикризисная специальная группа, куда входили представители всех родов войск и Министерства иностранных дел. Но советский внешнеполитический механизм оставался неизменным. Когда нужно было принять особо важное решение, Хрущев собирал расширенный состав с участием секретарей ЦК, представителей МИД и Министерства обороны.

Хрущев решил пригрозить США войной. В подготовленном письме он обвинял Кеннеди в том, что он поставил СССР "ультимативные условия" и отвергал "произвольные требования США". Рассматривая блокаду как акт агрессии, толкающий человечество к пучине мировой ракетно-ядерной войны, Хрущев заявил, что "советское правительство не может дать инструкции капитанам советских судов, следующих на Кубу, соблюдать предписания американских ВМС, блокирующих этот остров. Если американцы будут действовать подобным образом, - продолжал он, - мы будем тогда вынуждены со своей стороны предпринять меры, которые сочтем нужными и достаточными для того, чтобы оградить свои права".

В тот момент, когда Кеннеди подписывал заявление о блокаде, все МБР, оснащение боеголовками, находились на острове в состоянии боевой готовности. Теперь Хрущеву оставалась лишь ждать и надеяться, что страх войны заставит Кеннеди пойти на попятную.

\section{Глава 10}

Апогей кризиса пришёлся на 27 октября. В этот день советские зенитчики сбили американский самолёт-разведчик над Кубой, пилот погиб. Многие восприняли это происшествие как «первый выстрел последней войны». Фидель Кастро обратился к Хрущёву с просьбой нанести превентивный ядерный удар по США.

И американское, и советское руководство находилось в тупике: война казалась неизбежной, но брать на себя ответственность за уничтожение мира никто не хотел.

Спасением оказалось сообщение советского посла в США Анатолия Добрынина. В ночь с 27 на 28 октября он встретился с генеральным прокурором Соединённых Штатов и братом президента Робертом Кеннеди. Тот заявил послу, что США готовы дать гарантии безопасности Кубы и ликвидировать ракетные базы в Турции. Добрынин немедленно сообщил об этом в Москву.

Н.С. Хрущёв прислал следующий срочный ответ: «Соображения, которые Р. Кеннеди высказал по поручению президента, находят понимание в Москве. Сегодня же по радио будет дан ответ на послание президента от 27 октября, и этот ответ будет самый положительный. Главное, что беспокоит президента, — а именно вопрос о демонтаже ракетных баз на Кубе под международным контролем, — не встречает возражений и будет подробно освещен в послании Н.С. Хрущёва».

Ядерный кризис привлек международное внимание к отчаянному положению Кубы. Тот факт, что Советский Союз дошел до такого уровня в защите Кубы, указывает на степень угрозы американского империализма. Кастро надеется, что теперь он может рассчитывать на международную поддержку в преодолении американской экономической блокады.

Попытка Кастро сделать хорошую мину при плохой игре тем не менее не смягчила его гнев по отношению к Москве. Он считал, что Хрущев повел себя неправильно. Его злила не только суть дипломатических усилий Хрущева, но и то, что все было сделано без консультации с Гаваной.

Хрущев предотвратил войну с США. Но если ликвидация ракет на Кубе озлобила Кастро, то и вся стратегия Кремля в Карибском бассейне потерпела неудачу. 29 октября Хрущев решил направить Анастаса Микояна в Гавану, чтобы урегулировать отношения с кубинцами.

\section{Глава 11}

Кеннеди в своей следующей речи пояснил, что США не собираются напасть на Кубу. "Я заявляю, что мы не нападем, но отчетливо представляем себе, что Кастро наш недруг". Отвечая на обвинения Микояна по поводу облета кубинской территории, он сказал, что полеты имели целью лишь проверить выполнение договоренности Кеннеди - Хрущев. "Ясно, что множественные и частые полеты на низких высотах, - настаивал Микоян, - были просто хулиганством со стороны США, а нынешние полеты - тоже хулиганство, но только на большой высоте". Кеннеди подчеркнул, что редкие полеты не должны беспокоить Кастро. "США считают, что до тех пор пока у них не будет адекватных методов проверки, таковую надо осуществлять какими-то другими средствами. Я хочу, - добавил он, - удостовериться в том, что американский народ еще раз не одурачат". В конце встречи он выразил надежду, что Кастро поведет себя сдержанно и не будет провоцировать их. "Я, конечно, помню, о чем я писал Хрущеву", - закончил Кеннеди.

Несмотря на добрые заявления Президиума, после возвращения Микояна в Кремле царила атмосфера взаимных обвинений. На Президиуме обошли молчанием деятельность человека, которому поручили планировать и проводить операцию "Анадырь", - генерал-полковника Семена Иванова - одного из высших военных чинов советской армии. Вторым объектом обвинения были разведывательные службы Советского Союза, которые вызвали ложную тревогу, даже не раскрыв тайны Исполкома.

Сверхдержавы дали инструкции своим представителям в ООН окончательно оформить урегулирование кризиса. Эта последняя фаза продолжалась еще несколько недель; к тому времени, когда Микоян покинул Северную Америку, все основные проблемы за исключением одной были практически решены. Последнюю - ликвидацию на Кубе всего ядерного оружия - Советы решили самостоятельно. На Рождество 1962 года советский корабль спокойно отплыл из Гаваны, увозя на борту последние тактические боеголовки. Операция "Анадырь" закончилась.

\chapter{Часть III Последствия}

\section{Глава 1 Речь в Американском университете}

Возникла новая ситуация, при которой у сверхдержав появилась возможность несколько снизить напряженность отношений. Понятно, что идеологическая борьба между ними продолжалась. Хрущев санкционировал помощь Алжиру, считая, что в конечном итоге поможет Анголе. Кеннеди упорно защищал Южный Вьетнам. Однако и Москва и Вашингтон пытались найти мирное решение стоящих перед ними проблем. Хрущев старался приучить Кастро обходиться без советских войск на острове. Со своей стороны, Кеннеди ограничил акции саботажа на Кубе, надеясь руками кубинцев, настроенных оппозиционно к режиму Кастро, выполнять свои задачи, что, однако, представлялось маловероятным.

Речь Кеннеди демонстрировала терпимость США в отношении Советского Союза, и это отвечало стремлению Хрущева к осуществлению горячо поддерживаемого им принципа мирного сосуществования. Одно дело, когда Роберт Кеннеди рисовал президента как борца против милитаристов в Вашингтоне, и совсем другое, когда сам президент произнес публичную речь о необходимости взаимного уважения.

Кубинский ракетный кризис открыл глаза Хрущеву на опасность неконтролируемой гонки вооружений. Но именно Кеннеди в июне 1963 года побудил советского лидера согласиться с необходимостью соглашения по контролю над вооружениями, чего особенно желал Кеннеди. Хрущев не изменил своей позиции по инспекциям. Он по-прежнему считал их попыткой ЦРУ вести шпионаж внутри страны. Но он был готов согласиться на запрещение испытания в атмосфере, что не требовало никакой проверки. В апреле 1962 года Кеннеди предложил это, сделав последнюю попытку успокоить Пентагон, который настаивал на возобновлении испытаний.

\section{Глава 2}

Так начался короткий период ослабления напряженности между сверхдержавами, так называемая разрядка. Летом 1963 года помимо подписания договора об ограничении испытаний ядерного оружия обе страны открыли круглосуточную горячую линию, чтобы впредь избежать возникновения ситуации, подобно октябрьской 1962 года, когда экстренные сообщения передавались часами. Кеннеди вновь предложил совместный полет на Луну, который СССР снова отверг. Однако в духе улучшения отношений Москва впервые согласилась с тем, что договор по разоружению по крайней мере на какое-то время позволит сверхдержавам сохранить свои ядерные арсеналы. До этого СССР предлагал либо взаимное ядерное разоружение в качестве первого шага, либо ничего.

Кубинский ракетный кризис стал достоянием истории, однако благодаря ему и Кеннеди и Хрущев были готовы пойти на улучшение отношений. Хрущеву нужны были более предсказуемые отношения с Кеннеди, а перед Кеннеди открылись возможности соответствующим образом настроить общественное мнение, чтобы добиться одобрения своей внешней политики. Как и прежде, на горизонте маячил Кастро как потенциальное препятствие для достижения взаимопонимания двух сверхдержав.

\end{document}
